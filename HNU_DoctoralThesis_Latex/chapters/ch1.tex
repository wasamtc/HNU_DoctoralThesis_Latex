\chapter{绪\quad 论}

\section{研究背景和意义}
随着信息技术的迅猛发展,人工智能(Artificial Intelligence, AI)已成为推动社会进步和经济转型的核心驱动力。自20世纪50年代图灵提出“机器能否思考”的问题以来,人工智能经历了从符号主义、连接主义到深度学习的演变过程。特别是在过去十年,深度学习技术的突破性进展,使得AI在图像识别、自然语言处理、自动驾驶等领域取得了革命性成就。根据麦肯锡全球研究所的报告,人工智能有望在2030年前为全球经济贡献约13万亿美元的价值增长[1]。然而,AI的快速发展也对计算资源提出了更高的要求,尤其是大规模模型的训练和推理过程,需要高效的硬件支持和算法优化。

\begin{figure}
    \centering    \includegraphics{figures/hnu-logo.png}
    \caption{Caption标注}
    \label{fig:1}
\end{figure}

